\section{Periodic Boundary Conditions}
\REM{Sum convention is always in action}
\subsection{Direct Coordinates}
\REM{Direct coords: $\b R$}
Lattice point labels:
\begin{align}
\b l
= \begin{pmatrix}
l^1 \\ l^2 \\ l^3
\end{pmatrix},
\end{align}
\emph{primitive} lattice matrix
\begin{align}
\uunderline{a} = \left( \b a_1, \b a_2, \b a_3 \right)
= \begin{pmatrix}
a_1^x & a_2^x & a_3^x \\
a_1^y & a_2^y & a_3^y \\
a_1^z & a_2^z & a_3^z
\end{pmatrix}
\end{align}
lattice point coordinates:
\begin{align}
\b R_{\b l} = \uunderline{a} \cdot \b l = \sum_i l^i \b a_i
\end{align}
atomic coordinate
\begin{align}
\b R_{i, \b l} = \b R_{\b l} + \b R^0_{i} +  \b U_{i, \b l},
\label{eq:R_il}
\end{align}
with $\b l$ denoting the lattice point, $\b R^0_{i}$ denoting the 
equilibrium position of atom $i$ inside the unit cell and $\b U_{i, \b l}$ 
gives the deviation from the equilibrium position.

\subsection{Fractional Coordinates}
\nid
Define \emph{reciprocal lattice} $\uunderline{b}$:
\footnote{The vectors $\b b^i$ are \emph{dual vectors} to $\b a_i$. In the 
standard Cartesian basis, vectors with upper index are represented as 
\emph{row} vectors, whereas lower inidices correspond to \emph{column} vectors.}
\begin{align}
	\uunderline{b}
	= \begin{pmatrix}
	\b b^1 \\ \b b^2 \\ \b b^3
	\end{pmatrix}
	= \begin{pmatrix}
	b^1_x & b^1_y & b^1_z \\
	b^2_x & b^2_y & b^2_z \\
	b^3_x & b^3_y & b^3_z
	\end{pmatrix}
	,
\end{align}
with
$$ \b a_i \cdot \b b^j = \b b^j \cdot \b a_i = \delta_i^j $$
\REM{Up to a factor for $2 \pi$} \\
such that
\begin{align}
	\uunderline{b} \cdot \uunderline{a} = \id_{3 \times 3}
\end{align}
and
\begin{align}
	\uunderline{b} = \itp{\uunderline{a}}
	~.
\end{align}

In fractional (direct) coordinates denoted with a tilde in the 
following, we 
give the atomic positions as 
coefficients of the lattice $\uunderline{a}$:
\begin{align}
\b R_{i, \b l} 
=~& ~ \uunderline{a} \cdot \b r_{i, \b l} = r_{i, \b l}^k \b a_k \\
	\implies \b r_{i, \b l} =~& ~\uunderline{b} \cdot \b R_{i, \b l} \\
	\stackrel{\eqref{eq:R_il}}{=}&
	\b l + \uunderline{b} \cdot \left(\b R^0_{i} +  \b U_{i, \b l}\right) \\
	\equiv~& \b l + \b r_i^0 + \b u_{i, \b l}, 
\end{align}
where we have defined the fractional coordinates
\begin{align}
	\begin{split}
	\b r_i^0 \equiv & ~\uunderline{b} \cdot \b R^0_{i} \\
	\b u_{i, \b l} \equiv & ~\uunderline{b} \cdot \b U_{i, \b l}~.
	\end{split}
	\label{eq:ru_frac}
\end{align}

\subsection{Supercells}
A supercell is a structure defined by lattice vectors $\b A_i$ that are 
linear combinations of the primitive lattice vectors $\b a_i$:
\begin{align}
\begin{split}
	\b A_i =&~  \b a_j S^{j}_{~i} = \uunderline{a} \cdot \b S_1  \\
	\Leftrightarrow \left( \b A_1, \b A_2, \b A_3 \right)
	=&~ \left( \b a_1, \b a_2, \b a_3 \right) 
	\cdot \left( \b S_1, \b S_2, \b S_3 \right)\\
	\Leftrightarrow \uunderline{A} =&~ \uunderline{a} \cdot \uunderline{S}
	~.
\end{split}
\label{eq:superlattice1}
\end{align}
\subsubsection*{Row Vector Definition}
When the lattice is defined by row vectors
\begin{align}
	\b a^i &= (a^i_x, a^i_y, a^i_z) \\
	\Leftrightarrow \uunderline{a} &= 
	\begin{pmatrix}
	\b a^1 \\ \b a^2 \\ \b a^3
	\end{pmatrix}~,
\end{align}
the relations are transposed, and we have
\begin{align}
	\tp{\uunderline{A}} = \begin{pmatrix}
	\b A^1 \\ \b A^2 \\ \b A^3
	\end{pmatrix}
	= \begin{pmatrix}
	S^1_{~i} \b a^i \\ S^2_{~i} \b a^i \\ S^3_{~i} \b a^i
	\end{pmatrix}
	= \begin{pmatrix}
	\b S^1 \\ \b S^2 \\ \b S^3
	\end{pmatrix}
	\cdot
	\begin{pmatrix}
	\b a^1 \\ \b a^2 \\ \b a^3
	\end{pmatrix}
	=
	\tp{\uunderline{S}} \cdot \tp{\uunderline{a}}
	~,
	\label{eq:superlattice2}
\end{align}
as compared to \eqref{eq:superlattice1}.

\emph{Superlattice} points:
\begin{align}
\b L
= \begin{pmatrix}
L^1 \\ L^2 \\ L^3
\end{pmatrix},
\end{align}
and
\begin{align}
\b R_{\b L} = \uunderline{A} \cdot \b L = L^i \b A_i
= \b a_j S^j_{~i} L^i~.
\end{align}
It follows how to express a superlattice point from the primitive ones and vice 
versa:
\begin{align}
\begin{split}
	l^i &= S^i_{~j} L^j \\
	\Leftrightarrow \b l &= \uunderline{S} \cdot \b L \\
	\Leftrightarrow \b L &= \itp{\uunderline{S}} \cdot \b l~.
\end{split}
\end{align}

In the basis of superlattice vectors $\b A_i$, \emph{superfractional} 
coordinates are computed as
\REM{Notice the change $(i, \b l) \rightarrow (I, \b L)$}
\begin{subequations}
	\begin{align}
	\b R_{I, \b L} 
	=~& ~ \uunderline{A} \cdot \b r_{I, \b L} = r_{I, \b L}^k \b A_k \\
	\implies \b r_{I, \b L} =~& ~\uunderline{B} \cdot \b R_{I, \b L} \\
	\stackrel{\eqref{eq:R_il}}{=}&
	\b L + \uunderline{B} \cdot \left(\b R^0_{I} +  \b U_{I, \b L}\right) \\
	\equiv~& \b L + \b r_I^0 + \b u_{I, \b L}~.
	\end{align}
	\label{eq:Rfrac_full}
\end{subequations}

\REM{More clear notation needed.}

\subsection{Summing over Atoms}
To sum over all atoms of a periodic system, we need to perform the sum
\begin{align}
	\lim\limits_{\hat{l}_i \to \infty}~
	\sum_{i=1}^{N_\t{pc}} 
	\sum_{\b l = (0, 0, 0,)}^{(\hat{l}_1, \hat{l}_2, \hat{l}_3)}
\longrightarrow
\lim\limits_{\hat{L}_i \to \infty}~
	\sum_{I=1}^{N_\t{sc}} 
	\sum_{\b L = (0, 0, 0,)}^{(\hat{L}_1, \hat{L}_2, \hat{L}_3)}~,
	\label{eq:sumlp}
\end{align}
with $N_\t{pc}$ denoting the number of atom in the primitive unit cell, 
$$N_\t{sc} = N_\t{pc} \cdot \det \uunderline{S}$$ 
the number of atoms in the supercell, $\hat{l}_i$, and $\hat{L}_i$ upper bounds 
to range of (super-)lattice points covered.

\subsection{Finite Size -- Supercell Approximation}
\REM{The whole point of the supercell construction is to express the truncation 
of the limits in \eqref{eq:sumlp} by setting $\hat{L}_i = 0$, thus restricting 
to the trivial superlattice point $\b L = (0, 0, 0)$, and looking at 
$\b l$ fulfilling 
$$ 0 ~\leq~ L^i = (\itp{\uunderline{S}})^i_{~k} ~ l^k 
~<~ 1
\quad \forall i ~. $$}

The idea of the supercell approach is to work in the supercell only, so that 
only the trivial superlattice point $\b L = \b 0$ is considered. The $\b L$ in 
Eq.\:\ref{eq:Rfrac_full} can thus be dropped:
\begin{subequations}
	\begin{align}
	\b R_I &= \uunderline{A} \cdot \b r_I \\
	\Leftrightarrow \b r_I &= \uunderline{B} \cdot \b R_I \\
	&= \uunderline{B} \cdot \left(\b R^0_{I} +  \b U_{I}\right) \\
	&\equiv \b r_I^0 + \b u_{I}~
	\label{eq:Rfrac}
	\end{align}
\end{subequations}

\nid In order to represent positions in the supercell by fractional coordinates 
in the primitive lattice, the reciprocal lattice $\uunderline{b}$ has to be 
applied:
\begin{align*}
	\b r_{i, \b l} &= \uunderline{b} \cdot \b R_{I} \\
	&= \uunderline{b} \cdot \left(\b R^0_{I} +  \b U_{I}\right) \\
	&= \b l + \b r^0_{i} + \b u_{i, \b l}~.
\end{align*}
This suggest to relate
$$ I = (i, \b l)~. $$
When $i$ and $\b l$ are given, the respective $I$ is trivially constructed by
\begin{align}
	\b R^0_I = \b R^0_(i, \b l)~.
\end{align}
Conversely, when $I$ is given, $(i, \b l)$ can be found by constructing the 
fractional coordinate in the \emph{primitive} lattice and mapping to the 
primitive cell with the modulo function:
\begin{align}
	\b r^0_{i, \b l} &= \uunderline{b} \cdot \b R^0_{I} \\
	&= \b l + \b r^0_{i} \\
	\implies \quad
	\begin{split}
	r^{0, k}_{i} &= \left( \uunderline{b} \cdot \b R^0_{I} \right)^k \mod{1} \\
	\b l &= \b R^0_{I} - \b r^{0}_{i}~.
	\end{split}
\end{align}

\subsection{Hessian}
\nid Potential energy in a given supercell \mbox{configuration 
$\{\b R\}$}
%\emph{per supercell} 
is written as
\begin{align}
\begin{split}
E_\t{pot} (\{\b R\}) 
=&~ E (\{\b R^0\}) \\
&~+ \quad \sum_{I, \alpha} \frac{\del E}{\del U_I^\alpha} \: U_I^\alpha \\
&~+ \halb \sum_{I \alpha, J \beta} 
\frac{\del^2 E}{\del U_I^\alpha \del U_J^\beta} \:
U_I^\alpha \, U_J^\beta \\
&~+ \mathcal{O}(u^3)~, \nonumber
\end{split}
\end{align}
with $\alpha$, $\beta$ denoting the Cartesian components \mbox{($x$, $y$, $z$)} 
and $I,J$ labelling the atoms in the supercell.\footnote{
	$\d \b R = \d \b U$ since $\d \b L = 0$ and $\d \b R^0$.}

Atomic force on atom $\b R_I$ in a given configuration $\{\b R\}$ when 
retaining only the second order term:
\begin{align}
\begin{split}
\b F_I = - \frac{\del E_\t{pot} (\{\b R\})}{\del \b R_I}
=&~ - \sum_J \uunderline{H}^{IJ} \cdot \b U_J~,
\end{split}
\end{align}
with the Hessian
\begin{align}
\begin{split}
	H^{IJ}_{\alpha \beta} &\equiv 
	\frac{\del^2 E}{\del U_I^\alpha \del U_J^\beta}~,
	\label{eq:Hessian1}
\end{split}
\end{align}
or, in matrix form emphasizing the functional form
\begin{align}
\begin{split}
\uunderline{H} (\b R_I, \b R_J) &= 
\left(\b \nabla_{\b U_I} \otimes \b \nabla_{\b U_J}\right) E~.
\label{eq:Hessian1}
\end{split}
\end{align}

\subsection{Symmetries of the Hessian}
\subsubsection{Continuous Symmetries}
Translation, Rotation, Transposition.

\subsubsection{Finite Translation Symmetry}
\REM{This IS a space group property. Treat in one shot!}
In a periodic system, the Hessian does not change when the lattice is shifted 
by a lattice vector $\b R_{\b l}$:
\begin{align}
	\uunderline{H} (\b R_I, \b R_J) 
	= \uunderline{H} (\b R_I + \b R_{\b l}, \b R_J + \b R_{\b l})
	\quad \forall \b l~.
\end{align}
For $\b R_I$ with $I=(i, \b l)$, it is thus sufficient to consider atoms in the 
primitive unit cell only by applying the appropriate shift ($-\b R_l$), so that
\begin{align}
\uunderline{H} (\b R_{i, \b l}, \b R_{j, \b l'}) 
&= \uunderline{H} (\b R_{i}^0, \b R_{j, \b l'} - \b R_{\b l}) \\
&= \uunderline{H} (\b R_{i}^0, \b R_{j, \b l'-\b l}).
\end{align}

\REM{Full hessian:
	\begin{align}
		\uunderline{H} (\b l, \b l')
		&~=~ \,\uunderline{H} (\b l - \b l') \\
		\implies~
		\uunderline{H}_{[3NL \times 3NL]} &~\mapsto~ \uunderline{H}_{[3NL 
		\times 
		3N]}
	\end{align}
	with $L$ denoting the number of lattice points under consideration.
	This should happen during the symmetry reduction without any extra work. Of 
	course it makes sense to have this more explicit to be later able to pick 
	the $[3N \times 3N]$ sublocks in the Fourier transform to come by the 
	dynamical matrix.
}


