%\documentclass[a4paper,12pt]{scrartcl}
\documentclass[nofootinbib,preprintnumbers,amsmath,amssymb,twocolumn]{revtex4-1}
\usepackage[utf8]{inputenc}
\usepackage[english]{babel}
\usepackage{amsfonts,amsmath,amssymb}
\usepackage{graphicx, xcolor, dsfont}
\usepackage{url}
	
%	\usepackage[right]{eurosym}
\usepackage{typearea,enumitem}
	\areaset[0mm]{17cm}{25cm} % abstand vom rand, seitenbreite und höhe
%\setkomafont{sectioning}{\rmfamily\bfseries}
%	\setkomafont{section}{\Large}
%	\setkomafont{subsection}{\large}
%	\setkomafont{subsubsection}{\normalsize}

%	\setlist{noitemsep}
%	\pagestyle{empty}
%	\usepackage{upgreek}

% Colorbox
\usepackage{empheq,tcolorbox}
\tcbuselibrary{skins}
\newcommand{\examplebox}[2]{
  \vspace{1em}
  \begin{tcolorbox}[enhanced,colback=black!5!white,colframe=black,sharp corners,title={\textbf{#1}}]
    #2
  \end{tcolorbox}
  \vspace{1em}
}

% comments:
\definecolor{dblue}{RGB}{31,119,180}
\newcommand{\FK}[1]{\textcolor{dblue}{{\bf FK: #1 }}}

% Theorems
\newtheorem{theorem}{Lemma}

\renewcommand{\b}[1]{\boldsymbol{#1}}
\renewcommand{\t}[1]{\text{#1}}
\renewcommand{\vec}{\textbf{vec}}
\newcommand{\unvec}{\text{unvec}}
\newcommand{\nid}{\noindent}
%	\newcommand{\ts}{\textstyle}
%	\newcommand{\ds}{\displaystyle}
\newcommand{\im}{\text{i}}
%	\newcommand{\weglassen}[1]{}
\newcommand{\halb}{\frac{1}{2}}
\renewcommand{\d}{\,\text{d}}
\newcommand{\dd}[1]{\frac{\d}{\d {#1}}
\newcommand{\dt}{\frac{\d}{\d t}}}
\newcommand{\mat}[1]{\mathbf{#1}}
\newcommand{\op}[1]{\hat{#1}}
\newcommand{\M}{\phantom{-}}
\newcommand{\norm}[1]{\vert {#1} \vert}
\newcommand{\Norm}[1]{\left\lVert {#1} \right\rVert}
\newcommand{\bra}{\langle }  
\newcommand{\ket}[1]{{#1} \rangle }
\newcommand{\Bra}[1]{\langle {#1} \vert}
\newcommand{\Ket}[1]{\vert {#1} \rangle  }
\newcommand{\mean}[1]{\langle {#1} \rangle}
\newcommand{\D}[1]{{#1}^\dagger}
\newcommand{\fD}[1]{{#1}^{ \phantom{\dagger}}}
\newcommand{\Prime}[1]{{#1}^\prime}
\renewcommand{\P}[1]{{#1}^\prime}
\newcommand{\fP}[1]{{#1}^{\phantom{\prime}}}
\newcommand{\del}{\partial}
\newcommand{\scalar}[2]{\langle {#1}, {#2} \rangle}
\newcommand{\Tr}{\t{Tr}}
\newcommand{\Str}{\t{Str}}
\newcommand{\T}[1]{\tilde{{#1}}}
\newcommand{\fT}[1]{{#1} \vphantom{\T{ #1}}}
\newcommand{\uunderline}[1]{\underline{\underline{{#1}}}}
\newcommand{\id}{\mathds{1}}
\newcommand{\tops}[2]{\texorpdfstring{#1}{#2}}
\renewcommand{\H}{\mathcal{H}}
\newcommand{\bk}{\boldsymbol{k}{}}


\begin{document}
%\noindent
%Florian Knoop  \hfill \today \\[1em]
%\begin{center}
%  {\Large Fitting Force Constants}
%  \end{center}
%\vspace*{0.7cm}

\title{Interpolation Scheme for \emph{ab initio} Green Kubo Method}
\author{Florian Knoop}
\affiliation{Fritz Haber Institute of the Max Planck Society, Faradayweg 4--6, 
	14195 Berlin, Germany}


\date{\today}
\maketitle

\section{Atomic Positions in Supercells}
\nid
In a unit cell with $N$ atoms, we label each by 
\mbox{$i \in \{1 \ldots N \}$} 
and its positions by 
$\b r_i$. 
In a periodic system with lattice vectors 
$\{ \b a_1, \b a_2, \b a_3 \}$, we define \emph{lattice points} $\b L$ as
\begin{align}
	\b L = l^1 \b a_1 + l^2 \b a_2 + l^3 \b a_3~.
\end{align}

\nid
Atoms located at $\b r_i + \b L$ are images of atom $i$ in the unit cell shifted by a 
lattice point. We write
\begin{align}
\b R_I = \b r_i + \b L ~,
\label{def:R_I}
\end{align}
and introduce the combinded index 
\begin{align}
	I = (i, \b L)~,
	\label{def:I}
\end{align}
denoting the corresponding image in 
the unit cell and the respective lattice point.

We label the equilibrium positions of atoms in an infinite lattice by
\begin{align}
	\b R_I^0 = \b r_i^0 + \b L ~,
	\label{def:R_I0}
\end{align}
and the deviation from the equilibrium by
\begin{align}
	\b R_I - \b R_I^0 = \b U_I ~,
\label{def:u_i}
\end{align}
with equilibrium positions $\b R_i^0$ and displacement $\b u_i$ for $N$ 
atoms.

\section{Potential Energy Surface}
\nid
Expand the potential energy surface around the equilibrium configuration
$\{ \b R_I^0 \}$:
\begin{align}
\begin{split}
	E_\t{pot}( \{ \b R_I \}) = E_\t{pot}( \{ \b R_I^0 + \b U_I \}) \\
	= E_0 + \frac{\partial E}{\partial R_I^\alpha} U_I^\alpha
	+ \halb \frac{\partial^2 E}{\partial R_I^\alpha~\partial R_J^\beta} 
	~ U_I^\alpha U_J^\beta
	+ \cdots~,
\end{split}
\end{align}
with
\begin{align}
	\b F_I &= - \frac{\partial E}{\partial \b R_I}, \\
	\uunderline{\Phi}_{I,J}	&= \frac{\partial^2 E}{\partial \b R_I ~ \partial \b R_J}~.
\end{align}


\bibliographystyle{unsrt}
\bibliography{references}
\end{document}
